\documentclass[12pt]{article}
\usepackage{cedric}

\lhead{\textsc{Dl} 4 : Remise en route en algèbre linéaire}
\title{\vspace{-3em}\textsc{Dl} 4 : Remise en route en algèbre linéaire}
\date{\vspace{-2.5em}pour le jeudi 3 novembre \vspace{-.7em}}

\usetikzlibrary{matrix}
\usepackage{mathtools}

\begin{document}

\maketitle
\thispagestyle{fancy}


\begin{exercice}[Une base de $\R^2$]
  On considère les deux vecteurs $\V u = {3 \choose 2}$ et $\V v = {5 \choose 3}$.

  \begin{enumerate}
    \item Résoudre l'équation $\lambda \V u + \mu \V v = 0$, d'inconnues $\lambda, \mu \in \R$. \\ 
      Que peut-on en déduire sur la famille $(\V u, \V v)$ ?
    \item Résoudre l'équation $\lambda \V u + \mu \V v = {3 \choose 5}$, d'inconnues $\lambda, \mu \in \R$.
    \item Pour $x , y \in \R$, résoudre l'équation $\lambda \V u + \mu \V v = {x \choose y}$, d'inconnues $\lambda, \mu \in \R$. \\ 
      Que peut-on en déduire sur la famille $(\V u, \V v)$ ?
    \item Soit $P = 
        %{{{
        \begin{tikzpicture}[
                                >=stealth',
                                baseline=(matrice-2-1.base), 
                                every left delimiter/.style={xshift=1ex},
                                every right delimiter/.style={xshift=-1ex},
                           ]
          \matrix (matrice) [matrix of math nodes, left delimiter = {[}, right delimiter = {]}, 
          ]
          {
            \phantom{.} & \phantom{.} \\
            \V u        & \V v        \\
            \phantom{.} & \phantom{.} \\
          };

          \draw [->] (matrice-2-1.north) -- (matrice-1-1.north) ; 
          \draw [->] (matrice-2-1.south) -- (matrice-3-1.south) ; 

          \draw [->] (matrice-2-2.north) -- (matrice-1-2.north) ; 
          \draw [->] (matrice-2-2.south) -- (matrice-3-2.south) ; 
        \end{tikzpicture}
        %}}}
        = 
        \begin{bmatrix}
          3 & 5 \\ 
          2 & 3 \\ 
        \end{bmatrix}
        $. 
        Montrer que $P$ est inversible et calculer $P^{-1}$. \\
        Que constate-t-on ?
      \item On a représenté ci-dessous les deux vecteurs $\V u$ et $\V v$. \\ 
        Graphiquement, 
        à quoi voit-on qu'ils forment une 
        famille libre de $\R^2$ ?
        une base de $\R^2$ ?
  \end{enumerate}
\end{exercice}
\begin{center}
  \includegraphics{dessins/vecteurs.pdf}
\end{center}

\begin{exercice}[Deux sous-espaces vectoriels de $\R^n$ \hint{illustration page suivante}]
  Soit $F$ le sous-ensemble \hint{un \important{plan}} de $\R^3$ tel que $F = 
  \left \{
    \begin{pmatrix}
      x \\ y \\ z \\ 
    \end{pmatrix}
    \in \R^3
    %\vert 
    \text{ tels que }
    x + 2y + z = 0 
  \right \}
  $.
  \begin{enumerate}
    \item Vérifier que $F$ est un sous-espace vectoriel de $\R^3$. 
    \item Trouver une base de $F$.
      \\\phantom{a}\hfill\hint{on pourra appliquer l'algorithme du pivot de Gauss au \guillemets{système d'équations} $x + 2y + z = 0$}. 
    \item Soit $G = \Vect \left [ 
      \begin{pmatrix}
        1 \\ 1 \\ 0  \\ 
      \end{pmatrix}
      ,
      \begin{pmatrix}
        2 \\ 1 \\ 1  \\ 
      \end{pmatrix}
      \right ]
      $. 
      Trouver une équation du plan $G$. 
    \item Trouver une base de la droite $F \cap G$. 
  \end{enumerate}
\end{exercice}

\begin{center}
  \includegraphics{dessins/plans.pdf}
\end{center}

\begin{exercice}[Ma matrice $3 \times 3$ préférée]
  On étudie quelques propriétés de la matrice 
  $A = 
  \begin{bmatrix}
    0 & 1 & 1 \\ 
    1 & 0 & 1 \\ 
    1 & 1 & 0 \\ 
  \end{bmatrix}
  $
  \begin{enumerate}
    \item \important{Calcul des puissances de $A$}
      \begin{enumerate}
        \item 
          Montrer que l'on a 
          $
          \forall n \in \N, \ 
          A^n = 
          \begin{bmatrix}
            a_n & b_n & b_n \\ 
            b_n & a_n & b_n \\ 
            b_n & b_n & a_n \\ 
          \end{bmatrix}
          $
          avec les relations :
          $
          \begin{array}[t]{l@{}l@{}r}
            a_{n+1} & {} =     {} & 2 b_n    \\
            b_{n+1} & {} = a_n {} & {} + b_n \\
          \end{array}
          $
          %          $a_{n+1} = 2 b_n$
          %          $b_{n+1} = a_n + b_n$

        \item 
          Montrer que les suites définies par 
          $
          \begin{array}[t]{l@{}l@{}l}
            u_n & {} = a_n {} & {} - b_n   \\
            v_n & {} = a_n {} & {} + 2 b_n \\
          \end{array}
            $
          sont géométriques.

        \item Donner l'expression du terme général des suites $(u_n)$ et $(v_n)$.
        \item Montrer que $\forall n \in \N, a_n = \frac{1}{3} \left ( 2 u_n + v_n \right )$, et trouver $\lambda, \mu \in \R$ tels que 
          %$\forall n \in \N, 
          $b_n = \lambda u_n + \mu v_n$. 
        \item Conclure sur le terme général 

        \item Vérifier 
          $\forall n \in \N$, 
          %l'écriture : 
          que :
          $\displaystyle A^n = 
          \frac{2^n}{3} E
          +
          \frac{(-1)^n}{3} F
          $
          pour deux matrices $E$ et $F$ à détailler.

      \end{enumerate}
    \item \important{Inversion de $A$}
      \begin{enumerate}
        \item \label{polChar}Vérifier 
          $A^2 = A + 2 I_3$
        \item En déduire que 
          $A \frac{1}{2}(A - I_3) 
          = \frac{1}{2}(A - I_3) A = I_3$.
        \item En déduire que la matrice $A$ est inversible et donner l'expression de $A^{-1}$.
      \end{enumerate}
    \item \important{Réduction}
      Dans cette question, on utilise les notations suivantes : 
      \begin{center}
        $
        \V u = 
        \begin{pmatrix*}[r]
          1 \\ 1 \\ 1 \\ 
        \end{pmatrix*}
        \!
        ,
        \V v = 
        \begin{pmatrix*}[r]
          -1 \\ 1 \\ 0 \\ 
        \end{pmatrix*}
        \!
        ,
        \V w = 
        \begin{pmatrix*}[r]
          -1 \\ 0 \\ 1 \\ 
        \end{pmatrix*}
        \!
        ,
        P = 
        %{{{ matrice des vecteurs colonnes en tikz
        \begin{tikzpicture}[
            >=stealth',
                                baseline=(matrice-2-1.base), 
                                every left delimiter/.style={xshift=1ex},
                                every right delimiter/.style={xshift=-1ex},
                              ]
          \matrix (matrice) [matrix of math nodes, left delimiter = {[}, right delimiter = {]}, 
          ]
          {
            \phantom{.} & \phantom{.} & \phantom{.} \\ 
          \V u & \V v & \V w \\
          \phantom{.} & \phantom{.} & \phantom{.} \\ 
        };

          \draw [->] (matrice-2-1.north) -- (matrice-1-1.north) ; 
          \draw [->] (matrice-2-1.south) -- (matrice-3-1.south) ; 

          \draw [->] (matrice-2-2.north) -- (matrice-1-2.north) ; 
          \draw [->] (matrice-2-2.south) -- (matrice-3-2.south) ; 

          \draw [->] (matrice-2-3.north) -- (matrice-1-3.north) ; 
          \draw [->] (matrice-2-3.south) -- (matrice-3-3.south) ; 
        \end{tikzpicture}
        %}}}
        =
        \begin{bmatrix*}[r]
          1 & -1 & -1 \\ 
          1 &  1 &  0 \\ 
          1 &  0 &  1 \\ 
        \end{bmatrix*}
        $
        \hspace{-.7em}
        ,
        %        \hspace{-1em}
        %        et 
        %        \hspace{-1em}
        $D = 
        \begin{bmatrix*}[r]
          2 & 0  & 0  \\
          0 & -1 & 0  \\
          0 & 0  & -1 \\
        \end{bmatrix*}
        $
      \end{center}

      \begin{enumerate}
        \item Résoudre l'équation $x^2 = x + 2$, pour $x \in \R$. \hint{équation tirée de\ref{polChar}}
        \item %Soit $B = A - 2 I_3$.
          Montrer que $\ker(A - 2I_3) = \vect(\V u)$.
        \item %Soit $B = A - 2 I_3$.
          Montrer que $\ker(A + I_3) = \vect(\V v, \V w)$.
        \item Montrer que 
          $AP = PD$.
          \\\phantom{a}\hfill\hint{Comme la matrice $P$ est inversible, on a donc $A = PDP^{-1}$}

      \end{enumerate}

      %$A^2 = A + 2 I_3$
    \end{enumerate}
    %  $
    %  \begin{bmatrix}
    %    -2 & 1  & 1  \\
    %    1  & -2 & 1  \\
    %    1  & 1  & -2 \\
    %  \end{bmatrix}
    %  $
    %
    %



  \end{exercice}


\end{document}
