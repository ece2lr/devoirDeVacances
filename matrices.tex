\documentclass[12pt]{article}
\usepackage{nouvellesCommandes}
\usepackage{cedric}
\usepackage{fourier}
\DeclareMathAlphabet{\mathcal}{OMS}{cmsy}{m}{n}
\setlength\parindent{0pt}
\title
{
  \faireTitre[p=.50,h=5]{Carrés magiques, d'après \textsc{Hec 2014}}
}
\newcommand\Esp{\mathcal{E}}
\newcommand\C{\mathcal{C}}

%\lhead{}
%\rhead{T\up{ale} \textsc{Es} 1, lycée St-Ex}
\begin{document}

\maketitle
\thispagestyle{fancy}

\section{Notations}

\begin{description}[labelindent=0pt]
  \item [Espace des matrices]~ \\
On note \(\M_{3}\left( \R\right) \) l'espace vectoriel des matrices carrées d'ordre 3 à coefficients réels.
  \item [Base canonique]~ \\
    On rappelle que la base canonique, notée $\B$, de \(\M_{3}\left( \R\right)\) est formée des 9 matrices \smash{$(E_{i,j})_{\substack{i\in\entiers{1}{3} \\ j\in\entiers{1}{3}}}$},

    où  : \quad 
    \begin{enligneItemize}
      \item tous les coefficients la matrice $E_{i,j} \in \M_{3}(\R)$ sont nuls,
      \item sauf pour un coefficient 1, à l'intersection des \(i\)\up{ème} ligne et \(i\)\up{ème} colonne.

    \hfiller
    \smash
    {%
      \hint
      {
        on a $E_{1,2} = 
        \begin{bsmallmatrix}
          0 & 0 & 0 \\ 
          1 & 0 & 0 \\ 
          0 & 0 & 0 \\ 
        \end{bsmallmatrix}
        $, p. ex\up{le}.%
      }%
    }%
    \end{enligneItemize}

    \moinsLigne
  \item [Sommes] Pour toute matrice \(A=\left( a_{i,j}\right)\in\M_{3}\left( \R\right) \), on note : 

    \hfiller
    {
      \begin{tabular}{cccr}
        \(s_{1}(A) =\sum_{j=1}^{3}a_{1,j}~\), & \(s_{2}\left(A\right) =\sum_{j=1}^{3}a_{2,j}~\),  & \(s_{3}(A)=\sum_{j=1}^{3}a_{3,j}\) & \hint{somme des coefficients des lignes}   \\
        \(s_{4}(A) =\sum_{i=1}^{3}a_{i,1}~\), & \(s_{5}\left(A\right) =\sum_{i=1}^{3}a_{i,2}~\),  & \(s_{6}(A)=\sum_{i=1}^{3}a_{i,3}\) & \hint{somme des coefficients des colonnes} \\
        \(s_{7}(A) =\sum_{i=1}^{3}a_{i,i}~\), & \(s_{8}\left(A\right) =\sum_{i=1}^{3}a_{i,4-i}\), & \multicolumn{2}{r}{\hint{somme des coefficients des diagonales}} \\ 
      \end{tabular}
    }
\end{description}

\section{Objectif et notations}
\begin{enumerate}
  \item Soit \(\Esp\) l'ensemble des matrices \(A\in\M_{3}\left( \R\right) \) telles que \(s_{7}(A) =0\).
  \begin{enumerate}
    \item \label{baseDeEsp1}
      Calculer la valeur de $s_7$ sur chacune des 9 matrices de la base canonique :\quad \smash{$(E_{i,j})_{\substack{i\in\entiers{1}{3} \\ j\in\entiers{1}{3}}}$}.

      Parmi celles-ci, lesquelles appartiennent à $\Esp$ ? \quad \hint{il y en a 6 sur les 9}
    %\item Quelle est la dimension de \(\Esp\) ?
    \item \label{baseDeEsp2}Pour que
      $A(a) = 
      \begin{bsmallmatrix}
        a & 0 & 0 \\ 
        0 & 1 & 0 \\ 
        0 & 0 & 0 \\ 
      \end{bsmallmatrix}
      $
      et
      $B(b) = 
      \begin{bsmallmatrix}
        b & 0 & 0 \\ 
        0 & 0 & 0 \\ 
        0 & 0 & 1 \\ 
      \end{bsmallmatrix}
      $
      soient dans $\Esp$, combien doivent valoir $a$ et de $b$ ?
    \item Montrer que \(\Esp\)~est un sous-espace vectoriel de \(\M_{3}\left( \R\right) \).
    \item En s'aidant des questions \ref{baseDeEsp1} et \ref{baseDeEsp2}, construire une base de $\Esp$.
      \par \hint{Cette base sera formée de $8$ matrices.}
  \end{enumerate}
\end{enumerate}
  \begin{objetGauche}
    \qquad
    \hint
    {%
      ainsi, on a :
      \smash
      {%
        $
        f(A) =
          \begin{bsmallmatrix}
            \smash[b]{\strut}
            s_{1}(A)\\
            s_{2}(A)\\
            s_{3}(A)\\
            s_{4}(A)\\
            s_{5}(A)\\
            s_{6}(A)\\
            s_{7}(A)\\
            \smash[t]{\strut}
            s_{8}(A)\\
          \end{bsmallmatrix}
        \in \R^{8}
        $.%
      }%
    }%
    \finObjet
    On étudie l'application \(f : \M_{3}(\R) \to \R^{8}\), qui à toute matrice $A$, fait correspondre le vecteur  :  \quad
  %\left( %{{{
    %  s_{1}(A),
    %  s_{2}(A),
    %  s_{3}(A),
    %  s_{4}(A),
    %  s_{5}(A),
    %  s_{6}(A),
    %  s_{7}(A),
    %  s_{8}(A),
    %\right)
  %}}}
    $f(A) = \big(s_k(A)\big)_{k\in\entiers{1}{8}}$.
  %\left(  %{{{
    %s_{1}(A) ,s_{2}(A) ,s_{3}(A) ,s_{4}(A) ,s_{5}(A) ,s_{6}(A) ,s_{7}(A) ,s_{8}(A) 
    %\right) 
    %\in \R^{8}
    %\).
    %de \(\R^{8}\)
  %}}}
  \end{objetGauche}
  \moinsLigne
\begin{enumerate}[resume]
  \item
  \begin{enumerate}
    \item Montrer que l'application \(f\) est linéaire.
    \item \label{grosseMatrice}
      \begin{objetGauche}
      $
          \newcommand\lignePointilles
          {
            \mathrlap{{}\ldots{}}\hphantom{E_{11}} & \mathrlap{{}\ldots{}} \hphantom{E_{21}} & \mathrlap{{}\ldots{}} \hphantom{E_{31}} & 
            \mathrlap{{}\ldots{}}\hphantom{E_{12}} & \mathrlap{{}\ldots{}} \hphantom{E_{22}} & \mathrlap{{}\ldots{}} \hphantom{E_{32}} & 
            \mathrlap{{}\ldots{}}\hphantom{E_{13}} & \mathrlap{{}\ldots{}} \hphantom{E_{23}} & \mathrlap{{}\ldots{}} \hphantom{E_{33}} \\
          }
        \begin{array}{c@{}c}
          & 
          \begin{bmatrix}
            E_{11}& E_{21}& E_{31} & E_{12}& E_{22}& E_{32} & E_{13}& E_{23}& E_{33} \\
          \end{bmatrix} \\ 
          \begin{bmatrix}
            \vphantom{E_{1,1}\ldots} s_{1}(\cdot)\\
            \vphantom{E_{1,1}\ldots} s_{2}(\cdot)\\
            \vphantom{E_{1,1}\ldots} s_{3}(\cdot)\\
            \vphantom{E_{1,1}\ldots} s_{4}(\cdot)\\
            \vphantom{E_{1,1}\ldots} s_{5}(\cdot)\\
            \vphantom{E_{1,1}\ldots} s_{6}(\cdot)\\
            \vphantom{E_{1,1}\ldots} s_{7}(\cdot)\\
            \vphantom{E_{1,1}\ldots} s_{8}(\cdot)\\
          \end{bmatrix}
          & 
          \begin{bmatrix}
            \lignePointilles
            \lignePointilles
            \lignePointilles
            \lignePointilles
            \lignePointilles
            \lignePointilles
            \lignePointilles
            \lignePointilles
          \end{bmatrix}
        \end{array}
      $
        \finObjet
      Calculer, pour 
        \begin{itemize}
          \item les 9 matrices $(E_{i,j})_{\substack{i\in\entiers{1}{3} \\ j\in\entiers{1}{3}}}$
          \item les 8 valeurs $s_k(E_{i,j})$, avec $k\in\entiers{1}{8}$.
        \end{itemize}

      On donnera le résultat sous la forme du tableau ci-contre.
      \end{objetGauche}
  \end{enumerate}
    \pagebreak
    ~\\
    On note \(\C\) la base canonique de \(\R^{8}\).
  \begin{enumerate}[resume]
    \item 
      \begin{objetGauche}
        ~ \\
      \qquad où 
        \smash
        {%
      $F=
      \begin{bmatrix}
        1     & \cdot & \cdot & 1     & \cdot & \cdot & 1     & \cdot & \cdot \\
        \cdot & 1     & \cdot & \cdot & 1     & \cdot & \cdot & 1     & \cdot \\
        \cdot & \cdot & 1     & \cdot & \cdot & 1     & \cdot & \cdot & 1     \\
        1     & 1     & 1     & \cdot & \cdot & \cdot & \cdot & \cdot & \cdot \\
        \cdot & \cdot & \cdot & 1     & 1     & 1     & \cdot & \cdot & \cdot \\
        \cdot & \cdot & \cdot & \cdot & \cdot & \cdot & 1     & 1     & 1     \\
        1     & \cdot & \cdot & \cdot & 1     & \cdot & \cdot & \cdot & 1     \\
        \cdot & \cdot & 1     & \cdot & 1     & \cdot & 1     & \cdot & \cdot \\
      \end{bmatrix}
      $
        }%
        \finObjet
      À la lumière de la question\ref{grosseMatrice},
      justifier l'énoncé :

        \centerline
        {%
          \guillemets
          {%
            la matrice 
            de \(f\) dans les bases \(\B\) et \(\C\) est 
            \(F\) 
          }%
            ,
        }%

        \hfiller
        \hint{avec des \guillemets{$\cdot$} pour les \guillemets{$0$}}
      \end{objetGauche}
      %\moinsLigne[1.5]
    %\item 
    %  Écrire la matrice \(F\) de \(f\) dans les bases \(\B\) et \(\C\).
  \end{enumerate}
    \setcounter{enumi}{2}
  \item On note \(\G\) l'ensemble des matrices \(A\in\M_{3}\left( \R\right) \) telles que : \quad 
    \begin{align*}
      s_{1}(A) =s_{2}(A) =s_{3}(A) =s_{4}(A) =s_{5}(A) =s_{6}(A) =s_{7}(A) =s_{8}(A)
    \end{align*}

  \begin{enumerate}
    \item Montrer que \(\G\) est un sous-espace vectoriel de \(\M_{3}\left( \R\right) \).
  \end{enumerate}
On note \(\ker(f)\) le noyau de l'application linéaire \(f\). 
  \begin{enumerate}[resume]
    \item Soit $M \in \M_{3}(\R)$ une matrice quelconque.

      \begin{objetGauche}
        alors $f(M) = 0$. \hspace{3cm}~
        \finObjet
      Montrer que si : \quad
      \begin{enligneItemize}
        \item $M \in \G$, 
        \item et de plus, $s_{7}(M) = 0$,
      \end{enligneItemize}
      \end{objetGauche}
      %Montrer que si : \quad
      %\begin{enligneItemize}
      %  \item $M \in \G$, 
      %  \item et de plus, $s_{7}(M) = 0$,
      %\end{enligneItemize}
      %alors $f(M) = 0$.
      \moinsLigne[2]
    \item Montrer aussi la réciproque de l'énoncé précédent. 

      En déduire que : \quad \(\G\cap\Esp=\ker(f)\).
  \end{enumerate}
  \item 
On note \(J\) la matrice de \(\M_{3}\left( \R\right) \) dont tous les coefficients sont égaux à 1. 
    \hfiller
    \hint
    {
      Ainsi : \quad
      \smash
      {%
        $J = 
        \begin{bsmallmatrix}
          1 & 1 & 1 \\
          1 & 1 & 1 \\
          1 & 1 & 1 \\
        \end{bsmallmatrix}
        $.%
      }%
    }
  \begin{enumerate}
    \item Calculer $f(J)$.
  \end{enumerate}
    Soit $G$ une matrice quelconque telle que : \quad $G \in \G$.
  \begin{enumerate}[resume]
    \item Montrer que $f(G) = s_7(G) \cdot f(J)$.
    \item Pour que $G - t \cdot J$ soit dans $\ker(f)$, quelle doit être la valeur de $t$ ?
    \item En déduire que toute matrice de \(\G\) est la somme : \quad
      \begin{enligneItemize}
        \item d'une matrice de \(\ker(f)\), et
        \item d'une matrice de $\Vect(J)$.
      \end{enligneItemize}
      \moinsLigne[2]
      %de manière unique 
  \end{enumerate}
  \begin{enumerate}[resume]
    %\item Quel est le rang de l'application \(f\) ?
    \item Trouver une base de \(\ker(f)\).
  \end{enumerate}
\end{enumerate}

\end{document}
